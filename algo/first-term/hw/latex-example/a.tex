\documentclass{article}

\usepackage[utf8]{inputenc} % если ваш файл содержит русский текст, нужно указать кодировку
\usepackage[russian]{babel} % для того, чтобы писать русский текст
\usepackage{amsmath} % для команды equation*
\usepackage{hyperref} % для вставки ссылок
\usepackage{graphicx}
\title{My first document}
\date{2017-09-08}
\author{Jon Snow}

% до этого места была преамбула, тут можно задавать разные значения переменных, включать пакеты, а также указывать вещи, которые не суждено поменять посреди документа
\begin{document}
  \pagenumbering{gobble} % сейчас будет титульная страницу, отключим нумерацию страниц

  \maketitle % эта команда печатает титульную страницу
  \newpage % эта команда начинает новую страницу
  \pagenumbering{arabic} % включим нумерацию страниц обратно

  Hello World!

  \newpage

На этой странице все понятно из исходного кода. Вообще, интересный tutorial на английском можно почитать \href{https://www.latex-tutorial.com/tutorials/}{здесь}. Да, это была ссылка.

\section{Section}

Hello World!

\subsection{Subsection}

Structuring a document is easy!

\subsubsection{Subsubsection}

More text.

\paragraph{Paragraph}

Some more text.

\subparagraph{Subparagraph}

Even more text.

\section{Another section}

\newpage

Команда \texttt{equation} вводит формулу и приписывает справа номер:

\begin{equation}
  T(n) = \sqrt{n} T(\sqrt{n}) + n
\end{equation}

Для этой цели подключен пакет \texttt{amsmath}, команда \texttt{equation*} как \texttt{equation}, только не приписывает номер:

\begin{equation*}
  T(n) = \sqrt{n} T(\sqrt{n}) + n
\end{equation*}

Можно также без всяких \texttt{equation} делать формулу с помощью \texttt{\$\$}:

$$ T(n) = \mathcal{O}(n \log{\log{n}}) $$ 

А также можно показать, что $T(n) = \Omega(n \log{\log{n}})$. Иногда также полезно замечать, что $(A \implies B) \implies (\neg{B} \implies \neg{A})$. А как вы заметили здесь, формулы можно встраивать в текст с помощью одинарного \texttt{\$}.

Еще вы заметили, что я использовал \texttt{\textbackslash{}texttt}, чтобы писать \texttt{моноширинный текст}.

Чтобы переходить к следующему абзацу просто дважды переведите строку, а чтобы перевести строку введите \texttt{\textbackslash{}\textbackslash{}}.

Давайте вставим код:
\begin{verbatim}
  val a = IntArray(n, { scanner.nextInt() })
  for (i in 0 until n) {
    a[i] /= 2
  }
  a.sort()
  println("This is array: ${Arrays.toString(a)}")
\end{verbatim}


Давайте побольше разных формулок попробуем: $\sum\limits_{k=0}^n k^2 = \frac{n(n+1)(2n+1)}{6}$. Или: ${n \choose k} = \frac{n!}{k!(n-k)!}$. 

Отсортируем: $a_1, a_2, \ldots, a_n$, получим $a_{i_1}, a_{i_2}, \ldots, a_{i_n}$ такую, что $a_{i_j} \le a_{i_{j+1}}$ $\forall j: 1 \le j < n$. 

А если мы построим граф $G = \langle V, E \rangle$, и найдем в нем $v \to u$, где $v, u \in V$, а все ребра в множестве $E \subset V \times V$.

Наверное, вы захотите уметь вставлять картинку. Для этого вам понадобится \texttt{\textbackslash{}includegraphics}:

\includegraphics[scale=0.5]{itmo_logo_rus_vert_blue.eps}

Можно еще черно-белый попробовать с прозрачным фоном и в \texttt{.png}:

\includegraphics[scale=0.2]{bw_eng.png}

Ну и давайте таблицу еще сделаем и поместим ее по центру:

\begin{center}
  \begin{tabular}{|c|cc|} % с значит, что выравнивание по центру, вертикальная палка отвечает за то, где есть разделитель между столбцами
  \hline
    hello & it's & me \\
    QWERTY787788 & abacabadabacaba & ee \\
  \hline
    это & третья & строка \\
  \hline
  \end{tabular}
\end{center}


Перечислим что-нибудь:
\begin{enumerate}
  \item посмотреть на рекуррентную формулу, попробовать угадать ответ;
  \item если функция вызывается несколько раз, то может быть, стоит нарисовать дерево рекурсии;
  \item попробуем доказать оценку сверху ($\mathcal{O}$) по индукции;
  \item предположим, что $T(n) \le c \cdot f(n)$;
  \item ...
  \item PROFIT!!!!
\end{enumerate}

Еще можно точечки ставить:
\begin{itemize}
  \item третий
  \item первый
  \item второй
\end{itemize}

\end{document}
